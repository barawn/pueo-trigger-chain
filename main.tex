\documentclass{article}
\usepackage{graphicx} % Required for inserting images
\usepackage{amsmath}
\usepackage{tikz}
\usetikzlibrary{dsp,chains}

\title{PUEO trigger chain paper}
\author{P.S.~Allison}
\date{December 2025}

\begin{document}

\maketitle

\section{Introduction}
\section{Trigger filter chain}

The trigger system for PUEO is fully digital and contains multiple signal 
processing blocks to maximize the detection capability of the instrument,
as shown in Fig.~\ref{triggerfilterchain}. The filter chain consists of
a downsampling block, a matched filter, a CW rejection block, followed by an
upsampling block, and finally an automatic gain control and bit reduction
(AGC-BR) block before entering the digital beamformer. The CW rejection
portion of the filter chain consists of up to 2 programmable digital biquad
filters, with builds with 0, 1, and 2 biquads if needed for power reasons,
selectable during normal operation. Both the upsampling and downsampling
block are based on the same 33-tap finite impulse response halfband filter.

\subsection{Downsample/upsample blocks}

While significant power from the Askaryan signal is observed throughout the
band, the overall SNR peaks in the lower half. This therefore encourages
using a low-pass (halfband) filter to reduce the overall bandwidth to below
$750\,\text{MHz}$, which also allows the remainder of the trigger chain to
decimate by a factor of 2 and operate at $1500\,\text{MSa/s}$ to reduce
power. However, before the AGC-BR block, the signal is restored to
$3000\,\text{MSa/s}$ by zero padding followed by the same halfband filter,
allowing for finer beamforming delays.

The halfband filter is a 31-tap FIR filter operating at a supersample rate
of 8 with an additional 1 sample delay to align the center tap with the
original sample (at $z^{-16}$, two overall system clocks). The transfer
function is:
\[
\resizebox{\hsize}{!}{
H\left(z\right)=K
\left(
\begin{array}{cccccccc}
 & -23z^{-1} &  & +105z^{-3} &  & -263z^{-5} &  & +526z^{-7}\\
 & -949z^{-9} &  & +1672z^{-11} &  & -3216z^{-13} &  & +10342z^{-15}\\
+2^{14}z^{-16} & +10342z^{-17} &  & -3216z^{-19} &  & +1672z^{-21} &  & -949z^{-23}\\
 & +526z^{-25} &  & -263z^{-27} &  & +105z^{-29} &  & -23z^{-31}
\end{array}
\right)}
\]

where $K=2^{-15}$ for the downsample block and $K=2^{-14}$ for the
upsample block, resulting in a net unity gain for the combination of the
two. The symmetric nature of the filter lends itself to being
organized as the sum of two 4-tap systolic filters on individual samples,
with the preadd feature of the FPGA DSP block used to combine the samples
with common coefficients but reversed order, as shown in
Fig.~\ref{halfbandFilter}. As an example, ignoring pipeline registers,
for one of the filters, the first DSP takes in the sample at $z^{-3}$,
preadds the sample at $z^{-29}$, and outputs 
$105\left(z^{-3} + z^{-29}\right)z^{-8}$, as well as the sample at
$\left(z^{-11}\right)$ as a cascaded input. The next DSP again preadds
the same $z^{-29}$ sample, allowing it to compute
$1672\left(z^{-11} + z^{-21}\right)z^{-8}$ and add to the output of the
first DSP. The additional center tap value ($+2^{14}z^{-16}$) is simply
an upshifted value of the original input, and is added into one of the
systolic filters at the appropriate timepoint.

The same configuration is implemented for each of the 8 samples per
system clock. For the downsample block, since the output is decimated
afterwards, only the even samples are implemented. For the upsample
block, since the decimated odd inputs are zero-stuffed prior, only
the odd samples are implemented and the even samples (which only have
the center tap) are simply delayed to align to the odd samples.

The frequency response of this halfband filter is shown in
Fig.~\ref{halfbandFilter}. The limited rejection near Nyquist results in
some aliasing after decimation, but because the aliasing is equivalent
in all channels, the overall beamforming is not affected.

\begin{figure}[h]
\includegraphics[width=\textwidth]{plots/halfband_filter_resp.png}
\end{figure}

\subsection{Matched filter block}



\end{document}
