\documentclass{article}
\usepackage{graphicx} % Required for inserting images
\usepackage{subcaption}
\usepackage{amsmath}
\usepackage{tikz}
\usetikzlibrary{dsp,chains}
\DeclareMathAlphabet{\mathpzc}{OT1}{pzc}{m}{it}
\newcommand{\z}{\mathpzc{z}}

\title{PUEO trigger chain paper}
\author{P.S.~Allison}
\date{December 2025}

\begin{document}

\maketitle

\section{Introduction}
\section{Trigger filter chain}

The trigger system for PUEO is fully digital and contains multiple signal 
processing blocks to maximize the detection capability of the instrument,
as shown in Fig.~\ref{triggerfilterchain}. The filter chain consists of
a downsampling block, a matched filter, a CW rejection block, followed by an
upsampling block, and finally an automatic gain control and bit reduction
(AGC-BR) block before entering the digital beamformer. The CW rejection
portion of the filter chain consists of up to 2 programmable digital biquad
filters, with builds with 0, 1, and 2 biquads if needed for power reasons,
selectable during normal operation. Both the upsampling and downsampling
block are based on the same 33-tap finite impulse response halfband filter.

\subsection{Downsample/upsample blocks}

While significant power from the Askaryan signal is observed throughout the
band, the overall SNR peaks in the lower half. This therefore encourages
using a low-pass (halfband) filter to reduce the overall bandwidth to below
$750\,\text{MHz}$, which also allows the remainder of the trigger chain to
decimate by a factor of 2 and operate at $1500\,\text{MSa/s}$ to reduce
power. However, before the AGC-BR block, the signal is restored to
$3000\,\text{MSa/s}$ by zero padding followed by the same halfband filter,
allowing for finer beamforming delays.

The halfband filter is a 31-tap FIR filter operating at a supersample rate
of 8 with an additional 1 sample delay to align the center tap with the
original sample (at $z^{-16}$, two overall system clocks). The transfer
function is:
\[
\resizebox{\hsize}{!}{
H\left(z\right)=K
\left(
\begin{array}{cccccccc}
 & -23z^{-1} &  & +105z^{-3} &  & -263z^{-5} &  & +526z^{-7}\\
 & -949z^{-9} &  & +1672z^{-11} &  & -3216z^{-13} &  & +10342z^{-15}\\
+2^{14}z^{-16} & +10342z^{-17} &  & -3216z^{-19} &  & +1672z^{-21} &  & -949z^{-23}\\
 & +526z^{-25} &  & -263z^{-27} &  & +105z^{-29} &  & -23z^{-31}
\end{array}
\right)}
\]

where $K=2^{-15}$ for the downsample block and $K=2^{-14}$ for the
upsample block, resulting in a net unity gain for the combination of the
two. The symmetric nature of the filter lends itself to being
organized as the sum of two 4-tap systolic filters on individual samples,
with the preadd feature of the FPGA DSP block used to combine the samples
with common coefficients but reversed order, as shown in
Fig.~\ref{fig:halfbandStructure}, represented in supersample rate, where
$x[i]=z^{i}$ and system clocks generate a delay of $z^{-M}$ where
$M=8$.

As an example, ignoring pipeline registers, for one of the filters for
$y[7]$, the first DSP takes in $x[4]$ and preadds $x[2]z^{-3M}$. This
generates the $105\left(z^{-3}+z^{-29}\right)$ for $z^{7}$. The
next DSP receives $x[4]z^{-2M}$ in cascade input and the same $x[2]z^{-3M}$
generating the $1672\left(z^{-11}+z^{-21}\right)z^{-8}$ term for $z^{7}$,
which is added to the delayed output of the first DSP.
This same structure is used for the $x[6]$ and $x[0]$ terms, and the
outputs of those two chains are added together separately. Running the
two separate filters in parallel rather than serially requires an
additional adder, but was shown to reduce power by $\sim0.7\,\text{W}$
for all 8 channels due to the number of registers required to delay
inputs in the serial case.

The additional center tap value ($+2^{14}z^{-16}$) is simply
an upshifted value of the original input, and is added into one of the
systolic filters at the appropriate timepoint.

The same configuration is implemented for each of the 8 samples per
system clock. For the downsample block, since the output is decimated
afterwards, only the even samples are implemented. For the upsample
block, since the decimated odd inputs are zero-stuffed prior, only
the odd samples are implemented and the even samples (which only have
the center tap) are simply delayed to align to the odd samples.

The frequency response of this halfband filter is shown in
Fig.~\ref{fig:halfbandFreqResp}. The limited rejection near Nyquist
results in some aliasing after decimation, but because the aliasing is
equivalent in all channels, the overall beamforming is not affected.

\begin{figure}[h]
\begin{subfigure}{\textwidth}
\includegraphics[width=0.9\textwidth]{plots/halfband_filter_resp.png}
\caption{Frequency response}
\label{fig:halfbandFreqResp}
\end{subfigure}
% FIR filter as block diagram
\begin{subfigure}{\textwidth}
\begin{center}
\begin{tikzpicture}

	% Place nodes using a matrix
	\matrix (m1) [row sep=2.5mm, column sep=5mm]
	{
		%--------------------------------------------------------------------
		\node[dspnodeopen,dsp/label=above] (m00) {$x[4]$};    &
		\node[coordinate]                  (m01) {};          &
		\node[dspnodefull]                 (m02) {};          &
		\node[dspsquare]                   (m03) {$\z^{-2M}$}; &
		\node[dspnodefull]                 (m04) {};          &
		\node[dspsquare]                   (m05) {$\z^{-2M}$}; &
		\node[dspnodefull]                 (m06) {};          &
		\node[dspsquare]                   (m07) {$\z^{-2M}$}; &
		\node[coordinate]                  (m08) {};          &
		\node[coordinate]                  (m09) {};          &
		\node[coordinate]                  (m0X) {};          \\
        %--------------------------------------------------------------------
		\node[coordinate]                  (m40) {};          &
		\node[coordinate]                  (m41) {};          &
		\node[coordinate]                  (m42) {};          &
		\node[coordinate]                  (m43) {};          &
		\node[coordinate]                  (m44) {};          &
		\node[coordinate]                  (m45) {};          &
		\node[coordinate]                  (m46) {};          &
		\node[coordinate]                  (m47) {};          &
		\node[coordinate]                  (m48) {};          &
		\node[coordinate]                  (m49) {};          &
		\node[coordinate]                  (m4X) {};          \\
        %--------------------------------------------------------------------
		\node[dspnodeopen, dsp/label=above](m30) {$x[2]\z^{-3M}$};  &
		\node[coordinate]                  (m31) {};          &
		\node[dspadder]                    (m32) {};          &
		\node[dspnodeopen, dsp/label=above](m33) {$x[2]\z^{-3M}$};  &
		\node[dspadder]                    (m34) {};          &
		\node[dspnodeopen, dsp/label=above](m35) {$x[2]\z^{-3M}$};  &
		\node[dspadder]                    (m36) {};          &
		\node[dspnodeopen, dsp/label=above](m37) {$x[2]\z^{-3M}$};  &
        \node[dspadder]                    (m38) {};          &
		\node[coordinate]                  (m39) {};          &
        \node[coordinate]                  (m3X) {};          \\
		%--------------------------------------------------------------------
        \node[coordinate]                  (m10) {};          &
		\node[coordinate]                  (m11) {};          &
		\node[dspmixer, dsp/label=left]   (m12) {105};    &
		\node[coordinate]                  (m13) {};          &
		\node[dspmixer, dsp/label=left]   (m14) {1672};    &
		\node[coordinate]                  (m15) {};          &
		\node[dspmixer, dsp/label=left]   (m16) {-3216};    &
		\node[coordinate]                  (m17) {};          &
		\node[dspmixer, dsp/label=left]   (m18) {-263};    &
		\node[coordinate]                  (m19) {};          &
		\node[coordinate]                  (m1X) {};          \\
		%--------------------------------------------------------------------
		\\
		%--------------------------------------------------------------------
		\node[coordinate]                  (m20) {};          &
		\node[coordinate]                  (m21) {};          &
		\node[coordinate]                  (m22) {};          &
		\node[dspsquare]                   (m23) {$\z^{-M}$}; &
		\node[dspadder]                    (m24) {};          &
		\node[dspsquare]                   (m25) {$\z^{-M}$}; &
		\node[dspadder]                    (m26) {};          &
		\node[dspsquare]                   (m27) {$\z^{-M}$}; &
		\node[dspadder]                    (m28) {};          &
		\node[dspadder]                    (m29) {};          &
		\node[dspnodeopen,dsp/label=above] (m2X) {$y[n]$};    \\
		%--------------------------------------------------------------------
		\node[coordinate]                  (m50) {};          &
		\node[coordinate]                  (m51) {};          &
		\node[coordinate]                  (m52) {};          &
		\node[coordinate]                  (m53) {};          &
		\node[coordinate]                  (m54) {};          &
		\node[dspnodeopen, dsp/label=left] (m55) {$x[7]\z^{-2M}$}; &
		\node[coordinate]                  (m56) {};          &
		\node[coordinate]                  (m57) {};          &
		\node[coordinate]                  (m58) {};          &
		\node[coordinate]                  (m59) {};          &
		\node[coordinate]                  (m5X) {};          \\
		%--------------------------------------------------------------------
		\node[dspnodeopen,dsp/label=above] (m60) {$x[6]$};    &
		\node[coordinate]                  (m61) {};          &
		\node[dspnodefull]                 (m62) {};          &
		\node[dspsquare]                   (m63) {$\z^{-2M}$}; &
		\node[dspnodefull]                 (m64) {};          &
		\node[dspsquare]                   (m65) {$\z^{-2M}$}; &
		\node[dspnodefull]                 (m66) {};          &
		\node[dspsquare]                   (m67) {$\z^{-2M}$}; &
		\node[coordinate]                  (m68) {};          &
		\node[coordinate]                  (m69) {};          &
		\node[coordinate]                  (m6X) {};          \\
        %--------------------------------------------------------------------
		\node[coordinate]                  (m70) {};          &
		\node[coordinate]                  (m71) {};          &
		\node[coordinate]                  (m72) {};          &
		\node[coordinate]                  (m73) {};          &
		\node[coordinate]                  (m74) {};          &
		\node[coordinate]                  (m75) {};          &
		\node[coordinate]                  (m76) {};          &
		\node[coordinate]                  (m77) {};          &
		\node[coordinate]                  (m78) {};          &
		\node[coordinate]                  (m79) {};          &
		\node[coordinate]                  (m7X) {};          \\
        %--------------------------------------------------------------------
		\node[dspnodeopen, dsp/label=above](m80) {$x[0]\z^{-3M}$};  &
		\node[coordinate]                  (m81) {};          &
		\node[dspadder]                    (m82) {};          &
		\node[dspnodeopen, dsp/label=above](m83) {$x[0]\z^{-3M}$};  &
		\node[dspadder]                    (m84) {};          &
		\node[dspnodeopen, dsp/label=above](m85) {$x[0]\z^{-3M}$};  &
		\node[dspadder]                    (m86) {};          &
		\node[dspnodeopen, dsp/label=above](m87) {$x[0]\z^{-3M}$};  &
        \node[dspadder]                    (m88) {};          &
		\node[coordinate]                  (m89) {};          &
        \node[coordinate]                  (m8X) {};          \\
		%--------------------------------------------------------------------
        \node[coordinate]                  (m90) {};          &
		\node[coordinate]                  (m91) {};          &
		\node[dspmixer, dsp/label=left]   (m92) {-23};    &
		\node[coordinate]                  (m93) {};          &
		\node[dspmixer, dsp/label=left]   (m94) {-949};    &
		\node[coordinate]                  (m95) {};          &
		\node[dspmixer, dsp/label=left]   (m96) {10342};    &
		\node[coordinate]                  (m97) {};          &
		\node[dspmixer, dsp/label=left]   (m98) {526};    &
		\node[coordinate]                  (m99) {};          &
		\node[coordinate]                  (m9X) {};          \\
		%--------------------------------------------------------------------
		\node[coordinate]                  (mX0) {};          &
		\node[coordinate]                  (mX1) {};          &
		\node[coordinate]                  (mX2) {};          &
		\node[dspsquare]                   (mX3) {$\z^{-M}$}; &
		\node[dspadder]                    (mX4) {};          &
		\node[dspsquare]                   (mX5) {$\z^{-M}$}; &
		\node[dspadder]                    (mX6) {};          &
		\node[dspsquare]                   (mX7) {$\z^{-M}$}; &
		\node[dspadder]                    (mX8) {};          &
		\node[coordinate]                  (mX9) {};          &
		\node[coordinate]                  (mXX) {};          \\
	};

	% Draw connections
    \begin{scope}[start chain]
        \chainin (m55);
        \chainin (m56) [join=by dspline];
        \chainin (m26) [join=by dspflow];
    \end{scope}
	\begin{scope}[start chain]
        \chainin (m30);
        \chainin (m32) [join=by dspflow];
    \end{scope}
    \begin{scope}[start chain]
        \chainin (m80);
        \chainin (m82) [join=by dspflow];
    \end{scope}        
    \begin{scope}[start chain]
		\chainin (m00);
		\chainin (m02) [join=by dspflow];
        \chainin (m42) [join=by dspline];
        \chainin (m32) [join=by dspflow];
        \chainin (m12) [join=by dspconn];
		\chainin (m22) [join=by dspline];
	\end{scope}
	\begin{scope}[start chain]
		\chainin (m60);
		\chainin (m62) [join=by dspflow];
        \chainin (m72) [join=by dspline];
        \chainin (m82) [join=by dspflow];
        \chainin (m92) [join=by dspconn];
		\chainin (mX2) [join=by dspline];
	\end{scope}

	\foreach \i [evaluate = \i as \j using int(\i+1),
	             evaluate = \i as \k using int(\i+2),] in {2,4,6}
	{
        \begin{scope}[start chain]
            \chainin (m3\j);
            \chainin (m3\k) [join=by dspflow];
        \end{scope}
        \begin{scope}[start chain]
            \chainin (m8\j);
            \chainin (m8\k) [join=by dspflow];
        \end{scope}
		\begin{scope}[start chain]
			\chainin (m0\i);
			\chainin (m0\j) [join=by dspconn];
			\chainin (m0\k) [join=by dspline];
			\chainin (m3\k) [join=by dspconn];
			\chainin (m1\k) [join=by dspconn];
			\chainin (m2\k) [join=by dspconn];
		\end{scope}
		\begin{scope}[start chain]
			\chainin (m6\i);
			\chainin (m6\j) [join=by dspconn];
			\chainin (m6\k) [join=by dspline];
			% \chainin (m7\k) [join=by dspconn];
			\chainin (m8\k) [join=by dspconn];
			\chainin (m9\k) [join=by dspconn];
            \chainin (mX\k) [join=by dspconn];
		\end{scope}
        \begin{scope}[start chain]
            \chainin (m2\i);
            \chainin (m2\j) [join=by dspconn];
            \chainin (m2\k) [join=by dspconn];
        \end{scope}            
        \begin{scope}[start chain]
            \chainin (mX\i);
            \chainin (mX\j) [join=by dspconn];
            \chainin (mX\k) [join=by dspconn];
        \end{scope}
        \begin{scope}[start chain]
            \chainin (mX8);
            \chainin (mX9) [join=by dspline];
            \chainin (m29) [join=by dspconn];
        \end{scope}
	}

    \begin{scope}[start chain]
        \chainin (m28);
        \chainin (m29) [join=by dspflow];
        \chainin (m2X) [join=by dspflow];        
    \end{scope}
\end{tikzpicture}

\end{center}
\caption{Filter structure}
\label{fig:halfbandStructure}
\end{subfigure}
\caption{Halfband filter response and structure, omitting pipeline
registers. Only the structure for sample 7 is shown - the others are
identical except for index rotation and input delays.}
\label{fig:halfbandFilter}
\end{figure}

\subsection{Matched filter block}

The matched filter block mitigates the effect of the antenna and signal
processing chain on the SNR of the input signal. In order to implement a
power and resource-efficient matched filter, a multiplierless filter
was developed by exploiting the fact that the impulse response of the system
has an approximately exponential falloff in voltage. Therefore, the impulse
response was normalized to the peak value, and individual samples were
rounded to $0$, $\pm\frac{1}{4}$, $\pm\frac{1}{2}$, or $\pm1$.
A comparison of the normalized impulse response (post-halfband filter)
and the simplified impulse response in both the time and frequency
domain in shown in Fig.~\ref{matchedimpulse}. The reduced response is then
decimated by a factor of 2 to match the $1500~\text{MSa}/\text{s}$ sample
rate in the intermediate part of the processing chain,
time-reversed to create a matched filter, and then scaled by the nearest
power of 2 to a unity gain. The remaining gain variation is unimportant
as it is absorbed in the AGC-BR block.

\begin{figure}
    
\end{figure}

The transfer function of the matched filter at a sample rate of
$1500~\text{MSa}/\text{s}$ is given by

\begin{align*}
H\left(z\right) = \frac{1}{2^{5}} & (-1+z^{-3}-z^{-5}+z^{-7}+z^{-8} \\
                  & -z^{-9}-z^{-10}+z^{-11}+z^{-12}-z^{-13}   \\
                  & -2z^{-14}+2z^{-15}-4z^{-17}+4z^{-18}-2z^{-19}+z^{-20}) \\
\end{align*}

The filter is then simplified by pre-computing $P=\left(1-z^{-1}\right)$
for each sample. The computation of $P$ requires 4 adders at $M=4$, but
reduces the number of terms from $16M$ to $11M$, significantly
reducing the complexity of the filter.

\end{document}
